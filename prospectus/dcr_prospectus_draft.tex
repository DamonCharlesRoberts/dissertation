% ------- Create Preamble ------------
\documentclass [12pt]{article} 
\usepackage[a4paper]{geometry} 
\usepackage{amsmath, amsthm, amssymb, amsfonts}
 \usepackage{graphicx,epsfig}
\usepackage{booktabs} 
\usepackage{pslatex} 
\usepackage{caption} 
\usepackage{setspace} 
\usepackage{hyperref} 
\usepackage{multicol, multirow}
\usepackage{graphicx,epsfig}
\usepackage{booktabs}
\usepackage{pslatex}
\usepackage{caption}
\usepackage{setspace}
\usepackage{hyperref}
\usepackage{multicol}
\usepackage{textgreek}
\usepackage{pdfpages}
\usepackage{apacite}
\usepackage{floatrow}
\usepackage{natbib} % For references
\bibpunct{(}{)}{;}{a}{}{,} % Reference punctuation
\def\citeapos#1{\citeauthor{#1}'s (\citeyear{#1})}
\newtheorem{hypothesis}{Hypothesis}
\newtheorem{nullhypothesis}{Null Hypothesis}
\usepackage{xcolor}
\hypersetup{
    colorlinks,
    linkcolor={blue!50!blue},
    citecolor={blue!50!blue},
    urlcolor={blue!50!blue}
}

\usepackage{booktabs}
\usepackage{siunitx}
\newcolumntype{d}{S[input-symbols = ()]}


\usepackage{lscape}
\usepackage{tikz}
\usetikzlibrary{shapes.geometric, arrows}
\tikzstyle{startstop} = [rectangle, rounded corners, minimum width=3cm, minimum height=1cm,text centered, draw=black, fill=white]
\tikzstyle{io} = [rectangle, rounded corners, minimum width=3cm, minimum height=1cm,text centered, draw=black, fill=white]
\tikzstyle{io2} = [rectangle, rounded corners, minimum width=3cm, minimum height=1cm,text centered, draw=black, fill=white]
\tikzstyle{arrow} = [thick,->,>=stealth]
%---Set up author and title page
\singlespace
\title{Deliberative democracy and the emotional state}
\author{Damon C.\ Roberts\footnote{Ph.D Candidate,
Department of Political Science, University of Colorado Boulder, UCB 333, Boulder, CO 80309-0333. Damon.Roberts-1@colorado.edu.}}

%set up document
\date{\today}
\begin{document}
\maketitle


\newpage
\doublespace
\newpage
\section*{Introduction}
Throughout my time in graduate school, I continue to come back to questions involving our understanding of how well (or unwell) the American public is represented in politics. In doing so, I tend to examine how the public gets in the way of their expression of political voice. I started my undergraduate studies with the intention of earning a degree in Biology with the fuzzy goal of studying the human brain or attending medical school to work in neurosurgery. As I took classes on political psychology and political communication, I realized that I could understand how politics affects the brain and how features of the brain influence the way we engage in politics. I knew that I wanted to keep studying those questions for my dissertation.

The literature of affect in politics has a long intellectual history \citep[see][for a discussion]{marcus_2000_arps}. Many early discussions held a normative definition and interpretation of their role. Classic democratic theorists considered the role of the citizen to be that they rationally make decisions and exercise voice based on careful consideration of the facts. These normative interpretations of emotions' role in democratic politics lasted through much of the behavioral revolution in the social sciences, with grand theories of vote choice debating the rationality of the average voter. For the last $4$ or $5$ decades, political scientists leaning more on the literature in psychology shifted their tone. They began to consider emotions as serving functional purposes for memory encoding, retrieval, and its mediatory role in forming political attitudes. As these political psychologists have relied more heavily on the literature in psychology, the definitions and operationalization of effect began to demonstrate the relative lack of consensus about what emotions are and how they work. This mirrors the same challenges experienced by those who study affect in psychology and neuroscience \citep{sander_2013_chhan}. Even with the advent of more objective measures of emotion, definitions of what they are or theories about how they work still are somewhat splintered \citep{sander_2013_chhan}. The two primary ways scholars think of emotions in psychology have their roots in a debate between Charles Darwin, William James, and Walter Cannon. Darwin conceptualized emotions as autonomic reactions to a stimulus from the environment \citep{darwin_et-al_1998}. On the other hand, James and Cannon saw them as the interpretations of autonomic physiological reactions to a stimulus \citep{james_2013, cannon_1927_ajp}. In the study of affect in political science, the two primary theories that reign today are that emotions are either pre-conscious drivers of behaviors or that they are cognitive appraisals of political stimuli. 

Though many political psychologists study affect, there remain many open questions. First among them is how we define emotions. Though there are different interpretations, the literature seems to largely ignore the differences between the way the two major theories of affective politics define politics. Literature reviews on political affect are examples of the subfield's tendency to neglect this.

Relatedly, the literature has seemed to become its own subfield as there seems to be less borrowing from the literature in affective psychology and neuroscience. One implication of this second open question is that our research designs involving questions of affect rarely rely on multiple, simultaneous measures of emotion. Affective neuroscientists tend to rely on the familiar-to-affective-politics-scholars self-reporting of emotions from surveys and survey experiments, as well as neurological processes and physiological reactions. That is, affective neuroscience and psychology detect emotion through multiple measures. Many see emotions as just the neurological processes in both the pre-conscious and conscious brain, but not as the complete picture of the emotional state \citep{ralph_anderson_2018}. The other implication of a less interdisciplinary approach to the study of affect in politics is the loss of interest in clarifying the position one takes in the literature when choosing a particular definition of emotion. Consequently, we risk proposing theories that simultaneously suggest support for both affective intelligence and cognitive appraisal theories of emotion.

Besides questions of concept and measurement, there are many open substantive questions about the role of emotion. Though political psychologists took a more functional approach to study emotion, much of the work has been to understand the role of emotion from the perspective of attitude formation and political knowledge. As the behavioral revolution was underway, scholars of American political behavior characterized the public as unsophisticated partisans \citep{converse_1964}. This characterization continues today \citep[see][as examples]{delli-carpini_keeter_1996,achen_bartels_2016}. In attempts to understand the empirical variation implied by those claims, many have taken emotion as an angle to understand such questions. A smaller group of scholars have studied the role of emotion as motivation for forming behavior and not just attitudes. Interpersonal deliberation is one form of participatory behavior with a rich intellectual history.

Though there is work examining emotions' role in deliberation, the causal process remains somewhat unclear in this literature. In examining what motivates individuals to engage in deliberation in the first place, MacKuen and colleagues \citeyearpar{mackuen_et-al_2010_ajps} argue that anxiety encourages more willingness to engage in conversations with those they may disagree with. Using a different angle, Lyons and Sokhey \citeyearpar{lyons_sokhey_2014_pc} find evidence suggesting that emotion shapes the content of the conversation but cannot find consistent evidence that it motivated the conversation. In retrospectively evaluating a conversation, individuals who tend to report an enthusiastic mood toward politics tend to enjoy deliberating politics more, and those who are angry are more likely to avoid political conversations \citeyearpar{wolak_sokhey_2022_apr}. This work suggests that there is still quite a lot of uncertainty about the causal relationship between emotion and deliberation. Making a similar argument, Neblo \citeyearpar{neblo_2020_apsr} argues that emotion and deliberation are endogeneous. Emotion and deliberation each shape one another. However, as studying their relationship is a relatively small focus for the deliberation and emotion literature, there has yet to be a comprehensive detailing of the way in which emotion and deliberation cause and are caused by one another.

Previous theories that examine informal deliberation wholistically often rely on social psychological theories to motivate their claims. To date, the prevailing comprehensive theory of informal deliberation is rooted in social psychological theories. In a valuable and thorough analysis of the motivations of individuals finding themselves in political discussions, Carlson and Settle \citeyearpar{carlson_settle_2022_cup} argue that instrumental factors and those affecting social relationships explain individual heterogeneity in the willingness to engage in conversation, what they discuss when in the conversation, and the degree to which they engage in social distancing from outpartisans. While prevailing work primarily relied on assumptions about accuracy goals as a primary motivator for deliberation, Carlson and Settle \citeyearpar{carlson_settle_2022_cup} argued that there are also motivations rooted in self-image protection and partisan affiliation. In explaining the effects of deliberation, the comprehensive social psychological theory of informal deliberation suggests that individuals appraise their experience to update their networks and to examine their willingness to make similar choices in whether to engage in a conversation and how to engage in the conversation in the same way under similar circumstances. 

The focus of this dissertation is to propose a comprehensive cognitive theory of informal deliberation. Specifically, this dissertation relies on the Schacter-Singer theory of emotion \citep[see][for a discussion]{sander_2013_chhan}. Where one finds themselves stumbling into a conversation about politics, individuals will experience arousal - which often manifests through a physiological response \citep{carlson_settle_2022_cup, mutz_reeves_2005}. This physiological response is then evaluated consciously and assigned a label that reflects the valence of the emotion. This valence suggests to the individual which course of action is most appropriate to reduce the degree of arousal. This process continues throughout the experience in the conversation. At the close of the conversation, the theory of hot cognition would suggest that the valence and arousal one experiences at that time will lend itself to the memory of the event and will inform future reactions to similar stimuli \citep[see][]{taber_lodge_2006}.

Specifically, I examine the motivation to engage in a political conversation with the view that it arises from an affective response to a stimulus. Those who have strong physiological reactions to realizing a conversation they just walked into will take note of what that physiological reaction implies. What that physiological reaction implies falls on two dimensions: the magnitude of the difference between their current state and baseline state (arousal) and the directionality of that response (valence), whether it be inductive of anxiety, fear, anger, or enthusiasm, or joy. Once the level of arousal is determined, and the individual assigns a label for the emotion one is feeling, the individual has the choice of whether to find an excuse to leave quickly or to let it play out. As they hear more from others in the conversation and understand the topic of discussion, the online model of information processing would suggest that individuals would access information or encoded memory through recollection of the emotional state associated with that information \citep{taber_lodge_2006}. That then determines whether and how someone might jump in to express their point of view. As the conversation comes to a close, one takes stock of the experience that just occurred by assessing their emotional state throughout the process. Then that information informs one on how to encode whether that conversation was positive or negative, whether it is encouraging for whether or not to engage in future conversations with similar dynamics and on similar topics, or whether to take on some other form of politically relevant behavior such as a reluctance to talk to specific people in the future or about a particular topic, or even politics altogether.

The contribution, as I see it, is not just substantive but also hopefully clarifies the position that those studying affect in politics have taken from the perspective of those in other disciplines. As I will outline in the dissertation's first half, it should put perspective on the more significant debates that need to occur as we continue to examine emotion. While borrowing a different theory of emotion is not as popular among political scientists, it seeks to act as an alternative (at best) or at least play a devil's advocate role to stretch the validity of the two standard models used by political psychologists (at worst). With this goal, it meets another. The theory implies physiological, neurological, and cognitive measures of emotion as they operate in a distinct and ordered fashion.

IGNORE THE OUTLINE SECTION FOR NOW. MAKE SURE TO HAVE ARGUMENT NAILED DOWN. ONCE COMFORTABLE WITH THE DIRECTION, THEN THERE WILL BE MORE STRATEGIZING ON THE SPECIFICS OF TOPICS FOR THE CHAPTERS AND RESEARCH DESIGN.

\section*{Outline}
    \subsection*{Chapter 1}
How can we measure emotion and how can we do it during a political conversation? Thus far, self-reported measures of one's emotional state requires some degree of cognitive processing. That is, when one is asked how they are feeling or what emotions they are experiencing in response to a stimuli, this necessitates some degree of cognitive processing to evaluate one's emotional state. While I have few doubts these measures are relatively accurate, this measurement strategy limits our ability to detect emotion as deliberation is occurring.

As emotions are measured after a deliberation has occurred, we are likely missing a significant amount of information. Pre-treatment (or pre-conversation) measures of emotion probably do a good job at identifying the emotions present to motivate the conversation. However, with pre-and-post treatment measures of emotion, we are hard-pressed to make claims that these emotions are the ones felt through different stages of the conversation. These emotions are likely to be either emotional states at their end point. But we do not currently know whether these assumptions are indeed true and whether they have consequences for the claims we are making!

Furthermore, these measures on their own also are a very specific conceptualization of emotion. As the discussion in the introductory section highlights, the emotions we are identifying are likely to be better conceptualized as ``feelings''; which are only part-in-parcel of the emotional state. 

Using a combination of neurological, cognitive, and physiological measures of emotion, we not only will increase our capacity to detect variation in emotion over time, but we will also have a better picture of the emotional state as a whole. 

This first chapter is likely to do the most in terms of pushing the argument of measurement
        \subsubsection*{Research Design}

Won't get into this too much just yet until I have the argument nailed down a bit more.

    \subsection*{Chapter 2}
The motivation for this chapter will be based on following emotions throughout political conversation using the validated measure from chapter 1. It will examine how emotions motivate participation in a political conversation, how emotions are shaping the conversation's content, and how they shape retrospective evaluations of the conversations' purpose and value to the subjects.
        \subsubsection*{Argument}
        \subsubsection*{Research Design}
    \subsection*{Chapter 3}
The motivation for this chapter will examine the way that agreement and expertise contributes (or doesn't) to imitation. This imitation can be interpreted as the perceived emotion of a discussion partner and someone's desire to detect it, identify it, and to try to imitate it as part of a desire for belonging in the social group or to reduce conflict. The chapter will then discuss the conditions under which this does or does not work.
        \subsubsection*{Argument}
        \subsubsection*{Research Design}
\newpage
\bibliographystyle{apsr}
\bibliography{C:/Users/damon/Dropbox/bibliographies/representation/deliberation.bib,C:/Users/damon/Dropbox/bibliographies/information_processing/interest.bib,C:/Users/damon/Dropbox/bibliographies/information_processing/emotions.bib,C:/Users/damon/Dropbox/bibliographies/representation/attitudes.bib,C:/Users/damon/Dropbox/bibliographies/representation/knowledge.bib,C:/Users/damon/Dropbox/bibliographies/neuroscience_biopolitics/neuroscience/affective_neuroscience.bib,C:/Users/damon/Dropbox/bibliographies/representation/vote_choice.bib,C:/Users/damon/Dropbox/bibliographies/information_processing/motivated_reasoning.bib,C:/Users/damon/Dropbox/bibliographies/representation/trust.bib}
\end{document}