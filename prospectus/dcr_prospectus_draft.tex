% ------- Create Preamble ------------
\documentclass [12pt]{article} 
\usepackage[a4paper]{geometry} 
\usepackage{amsmath, amsthm, amssymb, amsfonts}
 \usepackage{graphicx,epsfig}
\usepackage{booktabs} 
\usepackage{pslatex} 
\usepackage{caption} 
\usepackage{setspace} 
\usepackage{hyperref} 
\usepackage{multicol, multirow}
\usepackage{graphicx,epsfig}
\usepackage{booktabs}
\usepackage{pslatex}
\usepackage{caption}
\usepackage{setspace}
\usepackage{hyperref}
\usepackage{multicol}
\usepackage{textgreek}
\usepackage{pdfpages}
\usepackage{apacite}
\usepackage{floatrow}
\usepackage{natbib} % For references
\bibpunct{(}{)}{;}{a}{}{,} % Reference punctuation
\def\citeapos#1{\citeauthor{#1}'s (\citeyear{#1})}
\newtheorem{hypothesis}{Hypothesis}
\newtheorem{nullhypothesis}{Null Hypothesis}
\usepackage{xcolor}
\hypersetup{
    colorlinks,
    linkcolor={blue!50!blue},
    citecolor={blue!50!blue},
    urlcolor={blue!50!blue}
}

\usepackage{booktabs}
\usepackage{siunitx}
\newcolumntype{d}{S[input-symbols = ()]}


\usepackage{lscape}
\usepackage{tikz}
\usetikzlibrary{shapes.geometric, arrows}
\tikzstyle{startstop} = [rectangle, rounded corners, minimum width=3cm, minimum height=1cm,text centered, draw=black, fill=white]
\tikzstyle{io} = [rectangle, rounded corners, minimum width=3cm, minimum height=1cm,text centered, draw=black, fill=white]
\tikzstyle{io2} = [rectangle, rounded corners, minimum width=3cm, minimum height=1cm,text centered, draw=black, fill=white]
\tikzstyle{arrow} = [thick,->,>=stealth]
%---Set up author and title page
\singlespace
\title{Deliberative democracy and the emotional state}
\author{Damon C.\ Roberts\footnote{Ph.D Candidate,
Department of Political Science, University of Colorado Boulder, UCB 333, Boulder, CO 80309-0333. Damon.Roberts-1@colorado.edu.}}

%set up document
\date{\today}
\begin{document}
\maketitle


\newpage
\doublespace
\newpage
\section*{Introduction}
Throughout my time in graduate school, I continue to come back to questions involving our understanding of how well (or unwell) the American public are represented in politics. In doing so, I tend to take the approach of examining how the public get in the way of their own expression of political voice. I started my undergraduate studies with the intention of earning a degree in Biology with the fuzzy goal of studying the human brain or attending medical school to work in neurosurgery. As I found myself taking classes on political psychology and political communication, I realized that I could understand the way that politics has an effect on the brain and how features of the brain influence the way that we engage in politics. I knew that for my dissertation, I wanted to keep studying those questions.

The literature of affect in politics has a long intellectual history \citep[see][for a discussion]{marcus_2000_arps}. Many early discussions held a normative definition and interpretation of their role. Classic democratic theorists considered the role of the citizen to be that they rationally make decisions and excersise voice based on a careful consideration of the facts. These normative interpretations of the role that emotions have in democratic politics lasted through much of the behavioral revolution in the social sciences with grand theories of vote choice debating the rationality of the average voter. For the last $4$ or $5$ decades, political scientists leaning more on the literature in psychology shifted their tone and began to consider emotions as serving functional purposes for memory encoding, retrieval, and its mediary role in forming political attitudes. As these political psychologists have relied more heavily on the literature in psychology, the definitions and operationalization of affect began to demonstrate the relative lack in consensus about what emotions are and how they work. This of course mirrors the same challenges experienced by those who study affect in psychology and neuroscience. Even with the advent of more objective measures of emotion, definitions of what they are or theories about how they work still are somewhat splintered. The two primary ways that emotions are thought to work in psychology have their roots in a debate between Charles Darwin and William James. Darwin conceptualized emotions as autonomic reactions to a stimulus from the environment. James, on the other hand, saw them as the interpretations of autonomic physiological reactions to a stimulus. In the study of affect in political science, the two primary theories that reign today are that emotions are either pre-conscious drivers of behaviors or that they are cognitive appraisals of political stimuli. 

Though many political psychologists study affect, there remain many open questions. First among them is how do we define emotions. Though there are different interpretations, the literature seems to largely ignore the gaping differences between the way that the two major theories of affective politics define politics. 

Relatedly, the literature has seemed to become its own subfield as there seems to be less borrowing from the literature in affective psychology and neuroscience. One implication of this second open question is that our research designs involving questions of affect rarely rely on multiple, simultaneous measures, of emotion. Affective neuroscientists tend to rely on the familiar-to-affective-politics-scholars self-report of emotions from surveys and survey experiments but also on neurological processes and physiological reactions. That is, affective neuroscience  and psychology detect emotion through multiple measures as many see emotions as just the neurological processes that happen in both pre-conscious and conscious brain, but not as the full picture of the emotional state \citep{ralph_anderson_2019}. The other implication of a less interdisciplinary approach to the study of affect in politics is the loss of interest in clarifying the position one takes in  the literature when choosing a particular definition of emotion. 

Besides questions of concept and measurement, there are many open substantive questions about the role of emotion. Though political psychologists took a more functional approach to studying emotion, much of the work has been to understand the role of emotion from the perspective of attitude formation and political knowledge. As the behavioral revolution was underway, scholars of American political behavior characterized the public as unsophisticated partisans. This characterization continues today. In attempts to understand empirical variation of those claims, many have taken emotion as a angle to understand such questions. A smaller group of scholars have taken to studying the role of emotion as motivation for forming behavior and not just attitudes. One such form of participatory behavior that has a rich intellectual history is deliberation. Despite the importance that democratic theorists place on deliberation for a functioning democracy, understanding the role that emotion plays in characterizing the motivations for and outcomes of deliberation is relatively weaker.

The focus of this dissertation will be to consider the role that emotions play on motivating one to participate in politics via deliberation and the way that they morph one's behavior during deliberation and the way they shape the resulting behaviors and attitudes from those political conversations. In doing so, I take on the criticism I've placed on the lack of curiosity among political psychologists to apply the debates from affective psychology and neuroscience to the study of affect in deliberative politics. 

The contribution, as I see it, is not just substantive but also that it hopefully clarifies the position that those studying affect in politics have taken from the perspective of those in other disciplines. As I will outline in the first half of the dissertation, it should put perspective on the larger debates that need to occur as we continue to examine emotion. The substantive contribution this dissertation intends to make is to follow the endogenous path that emotions have in understanding an individual's steps through deciding to engage in deliberation, how emotions effect the conversation and how conversation effects emotion, and how emotion shapes what one takes away from a conversation as well as connecting it to how the conversation shapes the mediary path that emotion plays in motivating more individual forms of participation. That is, at each step between deciding to participate in deliberatory politics, what occurs during political conversations, and the effects that political conversations have, I intend to examine the way that deliberation effects emotion and how emotion shapes deliberation.

\section*{Outline}
    \subsection*{Chapter 1}
        \subsubsection*{Argument}
        \subsubsection*{Research Design}
    \subsection*{Chapter 2}
        \subsubsection*{Argument}
        \subsubsection*{Research Design}
    \subsection*{Chapter 3}
        \subsubsection*{Argument}
        \subsubsection*{Research Design}
\newpage
\bibliographystyle{apsr}
\bibliography{}
\end{document}