\documentclass[12pt]{article}
\usepackage{hyperref}
\usepackage{setspace}
\usepackage[margin=1.2in]{geometry} 
\usepackage{amsmath}
\usepackage{tcolorbox}
\usepackage{amssymb}
\usepackage{amsthm}
\usepackage{lastpage}
\usepackage{fancyhdr}
\usepackage{accents}
\usepackage[english]{babel}
\usepackage[utf8x]{inputenc}
\usepackage{apacite}
\pagestyle{fancy}
\setlength{\headheight}{38pt}
\usepackage{natbib} % For references
\bibpunct{(}{)}{;}{a}{}{,} % Reference punctuation
\def\citeapos#1{\citeauthor{#1}'s (\citeyear{#1})}
\usepackage{xcolor}
\hypersetup{
    colorlinks,
    linkcolor={blue!50!blue},
    citecolor={blue!50!blue},
    urlcolor={blue!80!blue}
}


\renewcommand\qedsymbol{$\blacksquare$}

\newcommand{\ubar}[1]{\underaccent{\bar}{#1}}

\begin{document}

\rhead{Peeling back the onion: Affect produced by conversations is not like cake \\ Memo: Place in literature, motivation, and early thoughts on execution\\ Damon C. Roberts\\ \today}

\doublespacing
\section{Broadening the political scientists' view}

Those examining democratic outcomes often are concerned with the degree to which individuals participate in politics. If one is to be represented in democracy, one must show up \citep[see][]{griffin_newman_2005}. One way in which those interested in democratic theory study participation is through deliberation. There is a rich debate among these scholars examining whether deliberation with those around you increase the degree to which you participate in politics. Some argue that deliberation acts as a resource that one can rely on to participate in politics \citep{verba_et-al_1995}. Others argue that deliberation has conditional effects depending on whether this deliberation is with someone you disagree with \citep[see][]{mutz_2002}. The main point of this debate is whether someone becomes more engaged in politics or retracts from it as the result of deliberation. 

Another literature runs parallel to the social influence literature. Political psychologists examining the influence of political affect have conceptualized it as the degree to which one is inspired to become more engaged or to retract from it. There are two major ways that scholars of political affect think of emotion. The first heavily relies on subject self-reports to express whether a stimuli made them feel anxious, enthusiastic, or angry \citep{marcus_et-al_2006}. The other approach relies more on valence. Those who experience more positive reactions to stimuli encourage approach responses and those experiencing negative reactions to stimuli often are avoidant \citep[see][]{brader_marcus_2013}. This literature gives us predictions about what conditions we should expect people to participate in politics.

Our understanding of political affect is rather narrow as a result of our conceptualization of emotion. For the most part, we conceptualize emotion as a behavioral or cognitive manifestation of one's reaction to stimuli. Some of our best measures of political affect rely on self-reported feelings. Some scholars in the valence tradition have been using fewer cognitive-based measures to examine physiological responses to stimuli such as electromyography \citep{bakker_et-al_2020}, but this is still a relatively narrow view of how people respond to political stimuli as it sees emotion dichotomously and physiological reactions as a result of emotion. In a search using JSTOR's text mining tool, there are 22 published papers mentioning fMRI's in the JOP, AJPS, APSR, Political Behavior, Political Analysis, PSRM, JEPS, and APR. This search does not differentiate those that actually use an fMRI, so this is likely a high estimate.

Affective neuroscientists see emotions as a combination of the neurological, physiological, and cognitive reaction to a stimuli. Physiological responses to stimuli on their own are not clearly the physical manifestation of emotion. While Charles Darwin \citep{darwin_et-al_1998} conceptualizes physiological reactions to stimuli as the result of affect, Darwin's contemporaries have argued that autonomous physical reactions to a stimuli are what inform an individual to think about a stimuli \citep{james_2013}. The Cannon-Bard theory takes the view that emotion and physiological responses to stimuli are simultaneous \citep{cannon_1927}. That is, physiological responses to stimuli are not sufficient measures nor independent of emotion. Cognitive responses are also thought to be insufficient on their own as they are often a combination of more than just affect but a combination of affect and memory; for these scholars they are conceptualized as ``feelings" \citep{ralph_anderson_2018}, not as an emotional state like political psychologists tend to do. When we examine all features (neurological, physiological, and cognitive) together, we have a much clearer picture of one's emotional state \citep{ralph_anderson_2018}.

Not only do political psychologists have the tendency to measure feelings as opposed to the emotional state, but where there they focus on the neurological basis of emotion, political psychologists often attribute emotion to distinct regions of the brain. For example, in explaining the physiological differences between conservatives and liberals, political psychologists attribute larger amygdalas, not even the degree of activity within it, among conservatives to a higher propensity for system justification \citep{nam_et-al_2018} and for fear \citep{oxley_et-al_2008}. Affective neuroscientists today disagree with seeing emotions as attributable, or owned by, one region of the brain. Instead, affective neuroscientists argue that all emotions go throughout parts of the limbic system and prefrontal cortex. That is, an emotion is not characterized as the region of the brain that is active in response to a stimuli, but rather a distinct emotion is the particular network of neurons that are active in response to stimuli \citep{chang_et-al_2015}.\footnote{This is a rather robust literature. There is a consortorium which has combined the results of multiple fMRI studies to produce a meta-analysis which scholars in this discipline not only regularly contribute to, but compare their results for their individual studies to the meta analysis. This resource can be found at: https://neurosynth.org. By looking around on this site, you can see a number of emotions all active in different regions of the brain with some slight differences.}

We should take on a functional definition of emotions. Functional definitions of emotions in the neuroscience literature see emotion as a network among different regions in the brain; and view emotion as a combination of neurological, physiological, and cognitive reactions to stimuli \citep{ralph_anderson_2018}. Though, this functional definition often puts more emphasis on the neurological aspect due to the precision with which we can measure an emotional state and to not overinflate feeling with emotion \citep{ralph_anderson_2018}. Given the issues with how we conceptualize political affect, our weak measurement, and the traps we fall into by using oversimplified views of how emotion is encoded in the brain, it is pretty clear that those who study political affect should make a pivot to become more interdisciplinary. Furthermore, while there is significant work examining the degree to which social networks encourage engagement or disengagement with politics, our understanding of the role of affect is certainly limited despite its relevance.

To be clear, there are a few moves I want to make in this dissertation. First, to better tie the social influence literature with the role of emotions. There are clear links between the role of affect at encouraging aversion or more zest to participate in politics, and the tendency for the social influence literature to examine deliberation's ability to turn people on or off from politics. Moreover, participation is seen to be habit forming \citep{verba_et-al_1995}. As learning and memory has a strong connection with emotion \citep{ralph_anderson_2018}, an important component of tying together deliberation and affect would be to examine the role that emotion plays in a bayesian learning that forms the habitual political participant. The second move to make in this dissertation is to take a position similar to Ingrid Haas by advocating for an agenda on political neuroscience. The deep integration of political behavior with ``biopolitics" is largely glossed over by the discipline. This literature largely comprises studies done either by psychologists asking questions that political scientists see as having little value to adding to our discipline's open questions, as well as comprising of political scientists engaging in tentative interdisciplinary work. In characteristic naivete and ambition, I want this dissertation to deeply and meaningfully learn from psychologists and neuroscientists to allow for a degree of self-reflection.

\section{Gameplan}

\subsection{Tools for the experiments}
\begin{itemize}
    \item fMRI 
    \item EEG
    \item EMG
    \item Biomarker devices for hormone detection
    \item Self-reports
\end{itemize}

\subsection{Learning how to do these things}
\begin{itemize}
    \item Lots of online content. e.g. https://www.coursera.org/learn/functional-mri
    \item Nabbed some books (i.e. \textit{The Cambridge Handbook of Human Affective Neuroscience}) over the week that I hope to read once done with my last seminar paper (!) and grading.
\end{itemize}

\subsection{Timeline}
\begin{itemize}
\item[May, 2022 - ] Settle on some details:
    \begin{enumerate}
        \item Will this be a book?
        \item Who will be on the (prospectus) committee?
        \item What should be the focus of the different chapters?
    \end{enumerate}
\item[June $15^{th}$, 2022 - ] APSA Dissertation Improvement Grant  
\item[September, 2022 - ] Defend dissertation prospectus
\item[2023 - ] Experimenting, analyzing, and writing
\item[April, 2024 - ] If job market treats me well, finish and defend dissertation
\end{itemize}

\subsection{Potential Committee}
\begin{enumerate}
\item Anand Sokhey
\item Jennifer Wolak
\item Tor Wager
\item John Griffin/Taylor Carlson/Scott McClurg
\item Michael MacKuen/Gijs Schumaher/Bert Bakker - I met Gijs at MPSA and he along with the folks at the Hot Politics Lab are doing some really great stuff in this domain
\end{enumerate}

\newpage
\bibliographystyle{apsr}
\bibliography{/Users/damonroberts/Dropbox/bibliographies/neuroscience/affective_neuroscience.bib,/Users/damonroberts/Dropbox/bibliographies/representation/attitudes.bib,/Users/damonroberts/Dropbox/bibliographies/representation/participation.bib,/Users/damonroberts/Dropbox/bibliographies/information_processing/emotions.bib,/Users/damonroberts/Dropbox/bibliographies/institutions/institutional_theories/congress/policy_outcomes.bib}
\end{document}