\documentclass[12pt]{article}
\usepackage{hyperref}
\usepackage{setspace}
\usepackage[margin=1.2in]{geometry} 
\usepackage{amsmath}
\usepackage{tcolorbox}
\usepackage{amssymb}
\usepackage{amsthm}
\usepackage{lastpage}
\usepackage{fancyhdr}
\usepackage{accents}
\usepackage[english]{babel}
\usepackage[utf8x]{inputenc}
\usepackage{apacite}
\pagestyle{fancy}
\setlength{\headheight}{38pt}
\usepackage{natbib} % For references
\bibpunct{(}{)}{;}{a}{}{,} % Reference punctuation
\def\citeapos#1{\citeauthor{#1}'s (\citeyear{#1})}
\usepackage{xcolor}
\hypersetup{
    colorlinks,
    linkcolor={blue!50!blue},
    citecolor={black!50!black},
    urlcolor={blue!80!blue}
}


\renewcommand\qedsymbol{$\blacksquare$}

\newcommand{\ubar}[1]{\underaccent{\bar}{#1}}

\begin{document}

\lhead{Damon C. Roberts} 
\rhead{Dissertation \\ Memo: 30 Ideas \\ \today}

\doublespacing
\pagenumbering{roman}
\textbf{General}: Putting this memo together was a learning experience... hoo boy. Not only did I learn that I am really just asking tons of questions all about information: what it is (whether it is facts or norms), how its gathered, how it is disseminated, and how it is processed. From that, I noticed myself really getting excited about the idea of leaning more into being a political communication scholar or really leaning into trying to advance a research agenda that is really connected to neuroscience (don't worry... not genetics or just physiology; but cognitive processes) - which kind of fits my first, and abandoned, academic love of biology. I do have a few social psych angles that I think can be really fun. But, generally, I just really like the idea of seeing more work done on cognitive processes in political science - \textit{The Rationalizing Voter} is the book where I went from experimenting with the idea of graduate school and studying political psychology, to knowing that is what I wanted to do. One of my favorite components of it to this day is the rich theory derived from a deep knowledge of the cognitive psychology literature; I to this day think its implicit argument for political psychologists is that we need to continue our integration with psychology and neuroscience. I agree.

With this memo I tried to move away from being super general and just choosing topics like ``emotion" or ``white identity", but I \textit{tried} to come up with some possible questions I have jotted down in the past, or ones that have been rolling around in my head. They are kind of everywhere in terms of what literatures they engage most with, public opinion and surveys to something looking like not coming from a political science department. So this made it a bit tricky to meaningfully jump deep into the literature on a few of these ideas while compiling this memo. While it also helps save me a bit of time, I also wanted the questions I pose to be a bit rough as to make it a bit easier to refine as I think and talk more about the ideas. It also lets me have some room for adjusting when I finally pick a topic and dive in to write up a prospectus.

I also use a bit of jargon here, but I plan to definitely work on moving away from this in my writing of the prospectus and dissertation or at least do some introductions of what I mean when I have more space - so things are a bit more summarized than precisely laid out here.

I provide a star ranking for how excited I am about each of the ideas. It is out of 5 stars. Legend:
\begin{itemize}
\item $\star$ - I could live without putting this on the list; perhaps, there might be a really interesting angle that comes from brainstorming though, so I'm not going to shut it down.
\item $\star$ $\star$ - This has potential as an idea; I either am not in love with it because I am still unsure where exactly I want to go with this or I'm worried about how original or valuable it could be.
\item $\star$ $\star$ $\star$ - I generally like this idea. I just am unsure about it a little bit and want to do some more development. Or I may just not be in love with parts of the idea.
\item $\star$ $\star$ $\star$ $\star$ - Okay, I like this idea. I think I could have fun with this. 
\item $\star$ $\star$ $\star$ $\star$ $\star$ - I AM IN LOVE and I have not stopped thinking about this.
\end{itemize}
\tableofcontents
\newpage
\pagenumbering{arabic}
\setcounter{page}{1}
\cfoot{\thepage}

% Five Stars
\section{Five Stars}
\subsection{Affective Homeostasis and the reinforcement of participatory behaviors}
    \begin{itemize}
        \item \textbf{Question:} After an affective political stimuli encourages a particular behavior, does the degree to which the behavior encourage a return to affective homeostasis (may conceptualize this as relief) promote a learning process for behavior? Could it also be heterogeneous based on individual-level tendencies to respond to particular stimuli in a common way? What would this imply for a theory of affective homeostasis?
        \item \textbf{Bumper Sticker:} AIT and valence approaches to emotion both argue that there are these broader categories of emotions and that political stimuli encouraging particular emotive responses lead to predictable behavioral outcomes. For example, enthusiasm is a positive approach emotion and has a mobilizing effect. Once this political stimuli is gone and the effects of the emotion decay, probably as a result of the behavior, I think people have learned that when they feel enthusiasm from similar stimuli, they will try to engage in the same types of behaviors in the future. This might also be partially informed and may explain why some folks tend to vary in how they respond to the same political stimuli.
        \item \textbf{Sell:} Bigger picture I think this speaks to the participation literature in providing an explanation for how participation can be habit forming. I also think this speaks to the emotion literature in a few ways: examines duration of emotion's effects, examines whether behavior actually is what ends the effects of the stimuli or if it is the perception of action more generally that does it, AIT vs valence in terms of whether our conceptualizations of categories of emotions are too coarse or not nuanced enough. 
        \item \textbf{Chapter Outline:}
        \begin{enumerate}
            \item Posing the question; demonstrating how the existing literature takes a hard left turn from emotion $\rightarrow$ behavior $\rightarrow$ information processing rather than seeing the process through
            \item What is affective homeostasis? Is it relief? Is it the absence of affective stimuli?
            \item Does the behavior itself lead to the decay of emotion's effect or is it more about the recognition of taking some sort of action in response that leads to the eventual decay of emotion?
            \item What is the learning that happens when one takes a particular behavioral response to an emotion? What does it tell us if the behavior is successful? What does it tell us if the behavior is unsuccessful?
            \item Does messaging from political elites or a collective reaction encourage a particular learning process to get back to affective homeostasis?
            \item Is posting on social media an example of this process?
            \item Conclusions; what does this mean for how we think about emotion and political participation?
        \end{enumerate}
        \item \textbf{Challenges or shortcomings:} Need to do a lot to conceptualize ``affective homeostasis": is it relief? is it an emotion? is it the absence of emotion?; need to figure out what behaviors actually do to mitigate these political stimuli encouraging affective stimuli: is the action itself what moderates the effect of emotion? is it just more of a sense of taking some sort of action in response to the emotion that leads to its decay? is it just time? is it a function of time and the behavior? is it enough to just call it a mobilizing behavior or are there different types of mobilizing behavior and how do I know whether I can say someone will take the same behavior under different contexts?; How the hell do you make a research design for this? Do I get really psych with it? do I do some panel stuff (a la George Marcus and team)? can social media help? could I do a Lau and Redlawsk type of dynamic environment?
        \item \textbf{How excited am I about this idea?} $\star$ $\star$ $\star$ $\star$ $\star$
    \end{itemize}
\subsection{Whew: Relief as an important affective response to political stimuli}
    \begin{itemize}
        \item \textbf{Question:} What is relief? Is this a new type of affective response to political information/stimuli? How common is it? What is relief's functional role in hot cognition? Is it a trait or a state?
        \item \textbf{Bumper Sticker:} Relief is a positive affective response to political information that goes against common expectations for all political information to lead to negative reactions.
        \item \textbf{Sell:} Some really interesting recent work finds that congruent political messages do not lead to positive affect despite incongruent messages' effects on increasing negative affect. Some have concluded that as a result, we likely do not have much room for positive affect in politics. I disagree. I think this idea of ``relief" might be conceptualized as a positive emotion in politics. If we go out to detect this from a unconscious and cognitive approaches, I think there might be descriptively quite a bit of it. Further, I think it would highlight behavioral and attitudinal responses by those who have recently engaged with it: e.g. partisans who won a recent election they felt was high stakes. I know this is somewhat similar to the idea below, so maybe these ideas can be combined, but I can imagine spending quite a bit of time just trying to conceptualize and introduce this idea and then later having to provide a second project with a more nuanced discussion of its place in the causal chain.
        \item \textbf{Chapter Outline:}
        \begin{enumerate}
            \item Introduction; positive emotions and its lack of attention; its descriptive prevalence (or lack thereof)
            \item Conceptualizing relief; what is it; what does it mean;
            \item Detecting relief from both the cognitive and unconscious measurement approaches (e.g. physiological and self-reported measurement)
            \item Relief's place in hot cognition, opinion formation, and attitude change
            \item Conclusions; implications, and future directions
        \end{enumerate}
        \item \textbf{Challenges or shortcomings:} Would people care about this? Like some of my other ideas, I am going to need to get very comfortable with the substantive and methodological approaches of the folks in the building down the street - there are some more straightforward alternatives that don't require anything as hardcore as fMRI's and just muscle tension sensors and skin conductors that then use pooled time series approaches. But I think this could be rewarding and really fun and would be a fun research agenda that'll give me a distinct identity as a political psychologist - which could or could not pay off.
        \item \textbf{How excited am I about this idea?} $\star$ $\star$ $\star$ $\star$ $\star$ - I honestly think this could be combined with the previous one; but scope conditions might be a constraint here.
    \end{itemize}
\subsection{More than Anxiety, Fear, and Enthusiasm: Detecting and explaining the nuance in human emotion in political cognition}
    \begin{itemize}
        \item \textbf{Question:} While AIT is a significant improvement on valence approaches to understanding politically-relevant behaviors coming from emotion, should we be considering more than just these emotions?
        \item \textbf{Bumper Sticker:} We should be careful about only seeing emotions as important so long as they provide observable politically relevant behaviors. We should dig deeper than the social psychological utility of emotion and examine their more subtle impacts on things such as political cognition.
        \item \textbf{Sell:} While an ambitious undertaking, it could prove to help us move beyond the AIT framework of emotion. AIT has most of its utility in explaining its effects on participation and has been expanded to explain information processing in terms of motivated reasoning. However, I think this is a somewhat limited view of what emotions can tell us about information processing. I think if we are interested in examining political cognition, we need to move past social psychological approaches and take a more cognitive neuroscience perspective on emotions when looking at its impact on cognition.
        \item \textbf{Chapter Outline:}
        \begin{enumerate}
            \item Introduction; why should we take a more cognitive neuroscientific approach to examining emotion in political cognition?
            \item What does limiting ourselves to thinking about emotions as precursors to political participation do to our predictions of their explanatory utility of political outcomes?Descriptively, what emotions are there? How can examining more than the ones that are most useful at examining behaviors give us a more rich story about political information processing?
            \item Cognitive neuroscientific detection of emotion. In a more realistic way, how can political scientists use and develop computer vision models to detect these emotions during political information gathering tasks? What emotions are there in political information gathering tasks?
            \item Once detected, what do these emotions related to information gathering tell us about the different ways in which people process political information? How does this inform our theories and conclusions drawn from the online and memory based models of political cognition?
            \item Conclusions
        \end{enumerate}
        \item \textbf{Challenges or shortcomings:} Pffff. This would be a pretty huge undertaking in terms of funding, diving into a whole new field for understanding the substantive and methodological debates in cognitive neuroscience. I think this would be so cool, though. I think a movement for political psychologists to take a more neuroscientific approach could be so valuable. So, yeah, it is really ambitious, but this would definitely scratch that itch I have towards biology (started out as a biology major, was really involved with psychology, and wanted to do neurology in med school as an undergrad) and could keep me really engaged.
        \item \textbf{How excited am I about this idea?} $\star$ $\star$ $\star$ $\star$ $\star$
    \end{itemize}
% Four Stars
\section{Four Stars}
\subsection{A re-conceptualization of efficacy}
    \begin{itemize}
        \item \textbf{Question:} Is efficacy simply whether you feel that government has the ability to listen to you, or is it more about an expression of a feeling of alienation? What does this imply about existing measures of efficacy? Are they measuring multiple concepts?
        \item \textbf{Bumper Sticker:} Efficacy is not just a simple expression of whether you think that government is capable or chooses to listen to you, but it is also an expression of feelings of feeling alienated.
        \item \textbf{Sell:}The measure and conceptualization of efficacy has not really changed much since it was divided between internal and external efficacy. I think this measure might be problematic in that it could be a simple, more rational, sort of indication of ``yeah, government probably doesn't have the capability to listen to all of us" versus a more potent sense of alienation and exclusion from government.
        \item \textbf{Chapter Outline:}
        \begin{enumerate}
            \item History of how we have measured and conceptualized efficacy
            \item Explain what efficacy is, how it can be domain dependent, and how it the current conceptualization may represent two concepts.
            \item Conceptualizing and measuring alienation. What does it mean?
            \item Implications: Explaining thermostatic partisanship
            \item Implications: racial heterogeneity
        \end{enumerate}
        \item \textbf{Challenges or shortcomings:} I'd need to do a lot more careful thinking concepts, which is one of my weak points as I've come to find out. So this would be a bit of a challenge in terms of how precise I'd need to be in my thinking.
        \item \textbf{How excited am I about this idea?} $\star$ $\star$ $\star$ $\star$
    \end{itemize}
\subsection{Behavioral Polarization as identity norm conforming}
    \begin{itemize}
        \item \textbf{Question:} Are behavioral manifestations of polarization among the public learned behaviors from co-partisans?
        \item \textbf{Bumper Sticker:} Political elites are not just ideologically polarized but also affectively polarized. There are a number of observed behavioral manifestations of this polarization. Are the public just learning these behavioral norms from political elites as a way to be a good partisan?
        \item \textbf{Sell:} There is still \textit{some} contestation over whether partisanship among the public is ideological or identity based. In this debate, there is another question of whether the public learn this polarization from elites or whether the public are encouraging it. One potential way to examine the causal pathway here is to examine whether the public are learning these behavioral manifestations of polarization from political elites as they may be partisan norms. It would also be interesting to examine whether, once JQP learns these norms, whether they punish and reward co-and-out-partisans differently for participating in the same set of norms.
        \item \textbf{Chapter Outline:}
        \begin{enumerate}
            \item Documented polarization among elites and the public
            \item What are norms and how do they work in terms of communicating distinct social groups in politics
            \item Examining whether the public notice behavioral manifestations of polarization among elites
            \item If they recognize these norms, do they then mimic them to some extent?
            \item Do they reward or punish co-and-out-partisans for taking up these norms?
        \end{enumerate}
        \item \textbf{Challenges or shortcomings:} Is this just a paper? I've already drafted a research design and am kind of putting together a PAP on it when I find time. So this might just be good to do as a paper. Or I could do the paper, then book approach and just really go deep on this in the book treatment. Not sure. So advice would be much appreciated!
        \item \textbf{How excited am I about this idea?} $\star$ $\star$ $\star$ $\star$
    \end{itemize}
\subsection{The partisan brain: How elite communication shapes cognitive processes}
    \begin{itemize}
        \item \textbf{Question:} Does listening to a co-partisan follow similar (and as a result, strengthen) particular cognitive paths? Do out-partisan messages follow different paths? What about these paths are different? Where do they go? What parts of the brain are those things associated with?
        \item \textbf{Bumper Sticker:} Co-partisan and out-partisan communication tends to follow these re-enforcing cognitive pathways
        \item \textbf{Sell:} I want to build on the motivated reasoning literature by following the cognitive pathways that information coming from out-and-co-partisans follows. This would help with understanding not just polarization, but depending on how these pathways work and are reinforced, it can better help us understand the strength by which information just defaults to a particular cognitive process/path. It would speak to the extent to which you have to evoke an accuracy-based motivation for information processing.
        \item \textbf{Chapter Outline:}
        \begin{enumerate}
            \item Introduction; overview of online and memory based models of information processing
            \item Information encoding in the brain; hot cognition
            \item Tracing cognitive paths with brain imaging from elite messages by co and out-partisan elites
            \item Tracing cognitive paths with brain imaging from discussion partners or other members of the public
            \item Conclusions; implications
        \end{enumerate}
        \item \textbf{Challenges or shortcomings:} Okay, this is very neuroscience-y. This has a number of challenges in terms of the methodological and substantive techniques I'd need to learn to do this. I have some very basic understanding of this literature, but it'd definitely take quite of a bit of a step away from mainstream political science for me to go this route. This would also have some implications for research designs and types of funding I'd need to go after to execute these types of studies such as the need for fMRI scans. The Cognitive Psychology and Cognitive Neuroscience Department does have access to an fMRI machine, it looks like, through the Intermountain Neuroimaging Consortium. I have no clue whether I would be allowed to touch the thing, though. From \href{https://www.colorado.edu/mri/facilities-and-services}{this page}, it looks possible...
        \item \textbf{How excited am I about this idea?} $\star$ $\star$ $\star$ $\star$
    \end{itemize}
\subsection{Feeding off the audience: How audio and visual information from rallies teach partisans to be better partisans}
    \begin{itemize}
        \item \textbf{Question:} Do those who watch or listen to political events learn from co-partisan members of the public who attend these events? Do they adapt these similar sorts of behaviors or forms of vocal information to conform to it?
        \item \textbf{Bumper Sticker:} As an identity, partisanship contains a set of expectations about how to conform to the in-group and distinguish yourself from members of the out-group. For those who are not super ideological or strong partisans (e.g. because of relative lack of interest in politics, etc.), one source of information to make sure that you conform to these expectations is to mimic those who are part of your in-group both visually and audibly.0
        \item \textbf{Sell:} More knowledgeable, interested, and ideological folks tend to be stronger partisans. We are increasingly observing polarization among those that shouldn't be super polarized given our expectations from knowledge, interest, and ideological positions. One thing that might explain this is that the less politically engaged partisans may be learning how to be a proper partisan through observation of what it looks like to be a good partisan from those who attend political events - presumed to be stronger partisans than those who stay home. 
        \item \textbf{Chapter Outline:}
        \begin{enumerate}
            \item Introduction; what we know and who we expect to be really polarized; how does that contrast with what we actually observe; is it just a measurement issue?
            \item Provide a theory about how people might learn from co-partisans about how to behave as a partisan from those they think might be a stronger member of the group
            \item Experiments on whether watching and listening to co-partisans at political events predict self-reported affective polarization and politically relevant behaviors reflecting partisanship
            \item Do a lab experiment with some machine learning approaches with audio and video data to examine these actual behaviors in real time in response to these treatments
            \item conclusions and implications
        \end{enumerate}
        \item \textbf{Challenges or shortcomings:} Is the implication of this only that it explains why highly engaged and knowledgeable tend to be more knowledgeable and engaged? What would be the best frame that shows this idea is important besides my chance to play with some experiments and visual-based machine learning?
        \item \textbf{How excited am I about this idea} $\star$ $\star$ $\star$ $\star$
    \end{itemize}
\subsection{The effects of not taking overt racism seriously: How dismissing the alt-right effects participation among targeted racial groups}
    \begin{itemize}
        \item \textbf{Question:} What does language that dismisses the prevalence, potency, and saliency of alt-right groups do to participation among those racial groups who are targeted by the rhetoric of these alt-right groups?
        \item \textbf{Bumper Sticker:} This language dismissing the threat that these alt-right groups could be a form of deception and not lead to more or less political participation. If members of these targeted groups don't believe the dismissal, then it may increase turnout as a way to exert voice about it, or it can depress turnout if they believe it and no longer feel like they need to take action in response to it.
        \item \textbf{Sell:} With more media coverage and emboldened and very public demonstrations by explicitly racist alt-right groups, their intention is to spark fear among those that the groups express hatred against, to recruit those who might be seeking membership with those who might hold those views or be seeking a sense of belonging, what remains unclear is whether they have effects on political participation. There is a lot of minimization of these groups by calling them fringe, talking about how small of a minority they are in American society, and trying to continue to make them appear as uninfluential. The effects that they have on the psychology of those that are threatened by them, however, is likely not as small as their supposed influence is. Does seeing these very public and openly racist demonstrations by these groups increase or decrease turnout? Does it matter whether elite messaging about the group in either playing up the groups significance or downplaying it mediate these effects on participation?
        \item \textbf{Chapter Outline:}
        \begin{enumerate}
            \item Introduction; history of alt-right groups in US, their political influence, and their effects on the politics of those groups threatened by them
            \item Theory of what demonstrations by these groups do to influence political participation to those who are threatened by them; fill this out by explaining the mediating effect of messaging by political elites around these groups' demonstrations on the interpretation of what to do in response to these demonstrations.
            \item Do public and overtly racist demonstrations by alt-right groups depress political participation by those threatened by those groups?
            \item Do messages downplaying the influence these groups have mediate the relationship between these group's effects on participation?
            \item conclusions and implications
        \end{enumerate}
        \item \textbf{Challenges or shortcomings:} Sell. Both REP and Alt-Right stuff gets overlooked in the literature. I think the point of the paper is that we are doing that, and to our detriment, but I'd need to juggle a lot in terms of making my argument and convincing people that this really matters. I think comparativists would be more into it. :)
        \item \textbf{How excited am I about this idea?} $\star$ $\star$ $\star$ $\star$
    \end{itemize}
\subsection{Political chaos theory: how political volatility increases the public's reliance on heuristics}
    \begin{itemize}
        \item \textbf{Question:} Does volatility in the ``issue agenda's" focus increase incentives for JQP to rely on heuristics?
        \item \textbf{Bumper Sticker:} With a flood of news about political events, issues, and crises, facing not just domestic politics but also the global political landscape, this ever-changing set of debates and concerns encourages people to check out and rely on partisanship as a default for forming an opinion like we are socialized that every good citizen should have.
        \item \textbf{Sell:} Democratic theorists assumed of the American public as responsible, informed, and engaged citizens. Then, there was a correction in the literature which argued that the public were uninformed and not interested in being so. More recently, there have been corrections to this correction by arguing that while the public tend to be uninformed and not very well engaged, there is a lot of evidence pushing back to these more pessimistic views of JQP's ability to express their preferences to government. I think there is something to be said about the incentives that are created for the public where there is significant volatility in what issues are salient. If there is more volatility, this requires more cognitive effort and engagement to keep up with the new issue that society must address; to learn about the issue, to carefully consider ideological positions and where possible solutions fit in their ideological preferences, to develop and lobby for policy, etcetera. As the public tend to be disinclined to engage with politics in this way (e.g. part of the argument for representational versus direct democratic systems), where there is more volatility, the public are more likely to rely on heuristics. What we might expect, then, is that where there has been more issues government responds to, where news travels faster, where globalization is quite high, and where there seems to be yet another crisis that government needs to respond to by yesterday, motivated reasoning among the public might partly be explained by a learned base behavior of relying on partisanship to form judgments. That is, if we are inundated with a new issue and policy debates, but are also strapped for cognitive resources, we might just learn that the most efficient way of dealing with politics is to just default to whatever our co-partisans support. Simply, the chaos of politics today, might be teaching us to just check out. This is not reinventing any wheel; what the contribution, as I see it, of this argument is that it isn't just that macro-identity or motivated reasoning explain incentives to choose partisanship over accuracy, but additionally, it is also the information environment we live in. 
        \item \textbf{Chapter Outline:}
        \begin{enumerate}
            \item Introduction; what is political volatility, how has it been measured; what do the public know about politics; what incentives do the public have to become cognitively engaged with politics?
            \item What does political volatility do to encourage a reliance on heuristics to understand politics? What heuristics can they use and why is it partisanship? 
            \item Does it encourage people to rely more on prior beliefs because things are just ``too complex"?
            \item What does it do to political efficacy?
            \item What does it mean for what issues are salient and which ones remain salient (e.g. culture war issues)?
            \item What incentives are there for politicians to adopt strategies emphasizing complexity and chaos?
            \item Conclusions; implications
        \end{enumerate}
        \item \textbf{Challenges or shortcomings:} How do you measure issue volatility. Perhaps, it could be some sort of content tracing to examine what topics are getting significant coverage and then drop off, along with how many at the same time. Perhaps to get at this, we don't need an absolute measure of volatility but a chapter could be examining what the public think is a ``busy news day" and vary this to then examine at what point does exposure to all this information start to have little or negative effects on engagement and knowledge. I really have no idea whether people would think this idea packs enough of a punch as a dissertation. I think it could be a decent paper idea, but dissertation... not sure.
        \item \textbf{How excited am I about this idea?} $\star$ $\star$ $\star$ $\star$
    \end{itemize}
\subsection{Seeing is not believing: The limited effects that higher legislative representation have for those with more feminine traits on internal political efficacy and participation}
    \begin{itemize}
        \item \textbf{Question:} Does seeing more successful female candidates matter more for women in their early adult, politically transformative, years?
        \item \textbf{Bumper Sticker:} Not necessarily. As descriptive representation often has its limitations, one such form of limitation would be that most successful female political candidates tend to take on more masculine personalities. This means that for women who may hold some nascent political ambition, it may not convert to progressive ambition if they sense that those women are not like them - if they can't pinpoint the difference being more masculine gender personalities; which I feel like most voters might have a relatively difficult time articulating. What it might do, is actually deepen these gender discrepancies during the political socialization process.
        \item \textbf{Sell:} There is ample wonderful research out there examining the socialization of young women in politics. Often this work shows that not only are women socialized to be more feminine, but that politics is a masculine place and often do not get invitations to engage in these conversations. Some more recent work shows that it isn't about sex that leads to a gap in female representation but that it is a underrepresentation of those who have more feminine gender personalities who are underrepresented. This means that in recent elections that despite the increase in female representation, there is a lack of feminine representation. What remains unclear is whether that means the increase in female representation actually has positive cohort effects for young women who may not fully see themselves represented by these female politicians.
        \item \textbf{Chapter Outline:}
        \begin{enumerate}
            \item Introduction; the rise of the sex parity in Congress and the political socialization of young women
            \item Descriptive representation as a necessary but not sufficient condition for substantive representation
            \item Do young women feel more represented?
            \item Pinpointing the cause for the gaps in representation
            \item More politically efficacious and progressively ambitious, or more of the same?
            \item Conclusions and implications
        \end{enumerate}
        \item \textbf{Challenges or shortcomings:} I could be $100\%$ off on this bumper sticker. Would that throw the project off its rails if I am wrong? It would definitely force me to take a stance sort of challenging recent work that has come out in this area that I really like, so that wouldn't be a fun place for me. Seems a bit more like a paper rather than a dissertation.
        \item \textbf{How excited am I about this idea?} $\star$ $\star$ $\star$ $\star$
    \end{itemize}

% Three Stars
\section{Three Stars}
\subsection{Geographic Heterogeneity in Contemporaneous Conservatism}
    \begin{itemize}
        \item \textbf{Question:} Are rural and western conservatives distinct from the culture-war conservatives, what implications would this have for understanding ideological trends in the Republican party?
        \item \textbf{Bumper Sticker:} Rural and western conservatives are concerned with increasing interactions with regulatory agents from the Federal government. These conservatives are distinct from those who hold socially conservative views. Both, however, are part of the modern Republican party. The implication is that so long as the messages are framed around small government issues, there will likely be broad support among modern Republicans.
        \item \textbf{Sell:} There has been some work suggesting that western conservatives concerned with public land use were captured by Republican elites during the Obama era who were pushing against the ACA as a shared concern with growing federal government overreach. However, this work hasn't really examined how these conservatives became successful and loyal Republicans on other policy issues. I want to take a political communication approach to examine how conservative elites preaching conservative evangelicalism and other socially conservative views were able to convince these western conservatives that Democrats are universally and unambiguously associated with big government.
        \item \textbf{Chapter Outline:}
        \begin{enumerate}
            \item How has western and rural conservatism been different from social conservatism in the past? How has this evolved?
            \item How have Republicans taken up a ``small government" ideology?
            \item What are some features of the political communication during Obama's administration that indicate a Republican rallying cry of ``small government"?
            \item How did ``small welfare state" come to mean ``small government"?
        \end{enumerate}
        \item \textbf{Challenges or shortcomings:} I still need to think a bit more about this. I think some of the idea so far is pretty obvious and not all that punchy.
        \item \textbf{How excited am I about this idea?} $\star$ $\star$ $\star$
    \end{itemize}
\subsection{What are the origins of white political identity?}
    \begin{itemize}
        \item \textbf{Question:} Where does white identity come from and what is driving it?
        \item \textbf{Bumper Sticker:} White identity is likely not a result of economic trends but is more the result of race-based identity which desires for continued superiority - it is a desire to uphold the discriminatory status quo.
        \item \textbf{Sell:} While there have been descriptive treatments of white identity and a excellent examination of its implications, there still have been some debates as to whether the rise of white identity is the result of economic or identity-based concerns. It should help us understand and predict whether economic downturns will lead to this white backlash or whether this will be a new normal as the U.S. continues to trend towards a minority-majority country.
        \item \textbf{Chapter Outline:}
        \begin{enumerate}
            \item Perspective of economics and rational choice theories
            \item Perspective of social identity theory and discrimination
            \item Is it that people don't feel heard relative to the past?
            \item Is it distinct from partisanship?
            \item What emotions are involved in this? What implications does that have?
        \end{enumerate}
        \item \textbf{Challenges or shortcomings:} How do I distinguish myself from Jardina? Causal identification could be pretty tricky with observational data so I'd need to be relatively creative in terms of a research design.
        \item \textbf{How excited am I about this idea?} $\star$ $\star$ $\star$
    \end{itemize}
\subsection{The centrality of efficacy in a model of participation}
    \begin{itemize}
        \item \textbf{Question:} Can we think of the VBS model of participation as just a function of efficacy or self esteem?
        \item \textbf{Bumper Sticker:} Models of participation beyond the logistical challenges posed by institutional features seem to be all about the degree to which one is taught the importance of carrying out the civic duty of political involvement or whether they are mobilized (invited) to participate. How are these not exercises in increasing efficacy? Thinking of the Downsian model of participation, does it just all come down to a campaign's ability to convince them that it is not irrational to vote?
        \item \textbf{Sell:} Models of political participation have gone beyond the role of SES to argue that socialization and mobilization are significant predictors of participation. This has largely been around since the mid-90's. There seems to be a significant stall in the literature since this Beyond SES model was proposed. What remains unclear is whether socialization and mobilization really are just efforts at increasing efficacy. If so, this might indicate that our models of participation may just be the effect that these socialization and mobilization efforts have on sparking some sense of efficacy.
        \item \textbf{Chapter Outline:}
        \begin{enumerate}
            \item SES and Beyond SES model of efficacy
            \item Does socialization and mobilization increase political efficacy? Are these the only ways that one can increase efficacy and thus increase participation? That is, is it a moderating or mediating relationship?
            \item Using predictive models of participation to compare the value of these differing models of participation.
        \end{enumerate}
        \item \textbf{Challenges or shortcomings:} In VBS's model, efficacy does seem to be part of the story. So this isn't all too new. What would be the contribution would be examining whether these ``invitations" to participate provide a mediating or moderating effect on partisanship. That is, is the model of participation simply mobilization and socialization $\rightarrow$ efficacy $\rightarrow$ participation or is it mobilization and socialization $\times$ efficacy $\rightarrow$ participation
        \item \textbf{How excited am I about this idea?} $\star$ $\star$ $\star$
    \end{itemize}
\subsection{Bounds of accuracy in use of subtle party stereotypes in predicting candidate partisanship in non-partisan, low-information, settings}
    \begin{itemize}
        \item \textbf{Question:} To what extent can the public accurately use subtle party stereotypes to accurately classify a political candidate as a Democrat or Republican when in a non-partisan, low-information, setting?
        \item \textbf{Bumper Sticker:} The public definitely do not need party labels next to a candidate's name to pick up on it
        \item \textbf{Sell:} While we know that partisanship is a really important social force, there still is a lot to be done in understanding just how far to people rely on cues and stereotypes to make inferences about the partisanship of another person or candidate. Without explicit information of a party label, what other stereotypes do people pick up on to guess partisanship and what are the bounds of it?
        \item \textbf{Chapter Outline:}
        \begin{enumerate}
            \item Introduction; what we know about party stereotypes, cues, and subtle information used to identify members of the group
            \item Providing a theory about the importance of cultural stereotypes, personality traits, and campaign decisions in helping individuals to make inferences about a candidate's partisanship in a non-partisan, relatively low information environment
            \item Detecting the bounds of what cultural stereotypes (e.g. where they live, job that they have, education) can be accurately attributed to a candidate's partisanship
            \item Detecting the bounds of what campaign advertising(e.g. campaign advertisements, yard signs) can be accurately attributed to a candidate's partisanship
            \item Detecting the bounds of adherence to stereotyped group personalities (e.g. individualism, reactive to fear-inducing stimuli)
            \item Conclusions and implications
        \end{enumerate}
        \item \textbf{Challenges or shortcomings:} The research design for this would be... challenging. The amount of funds that would be needed to run enough experiments to test different levels of stimuli would be pretty significant - hello NSF.
        \item \textbf{How excited am I about this idea?} $\star$ $\star$ $\star$ - has potential, but as it stands in this memo is a bit meh IMHO.
    \end{itemize}
\subsection{Latino campaign strategy: How highlighting group membership is dangerous}
    \begin{itemize}
        \item \textbf{Question:} Does highlighting one's membership to the Latino community work out for Latino candidates?
        \item \textbf{Bumper Sticker:} Not really. It will encourage negative reactions among the racially resentful and it will not really have much of a positive or negative effect among Latino voters. That is, often times they know if you are Latino... so is it then just that it looks like you are pandering to white progressives? Does that cause some negative perceptions about whether you are just using the identity or are claiming membership with it?
        \item \textbf{Sell:} Studying Latino politics is pretty complicated as they are culturally and politically heterogeneous. There seems to not be tons of work examining how much expressing group affiliation with other Latinos tends to work as a campaign strategy. Among Black Americans, commitment to the community is quite important - but there is a recognition of need to cater to White voters to some degree. With Latinos, without this same heterogeneity in group preferences, it does not seem to be a viable strategy to ``prove" commitment to the group. It can also spark backlash among Whites who hold racially resentful attitudes.
        \item \textbf{Chapter Outline:}
        \begin{enumerate}
            \item Introduction; real world examples of pandering to Latinos; what we know about White and Latino political attitudes and preferences toward policy relevant to the community
            \item Provide a theory of the lack of success a Latino candidate may run into if they try to make their affiliation with the community salient in a campaign
            \item Reactions among Latinos to the salience of group attachments; might be variation between immigrant, $1^{st}$ generation, and $2^{nd} +$ generation Latinos
            \item Reactions among Whites to the salience of group attachments
            \item Conclusions and Implications
        \end{enumerate}
        \item \textbf{Challenges or shortcomings:} Latino ID is complicated as it is an umbrella category that flattens a number of diverse sub-identities such as Mexican, Brazilian, Cuban, etc. Maybe use \textit{Diversity's Child} as a template for how to deal with these umbrella identities? The problem is that Cuban vs. Mexican to Latino is not like Black vs. Latino to PoC...
        \item \textbf{How excited am I about this idea?} $\star$ $\star$ $\star$
    \end{itemize}
\subsection{Thanks a lot Gerry: How Gerrymandering Increases Political Polarization}
    \begin{itemize}
        \item \textbf{Question:} What are the affective effects of gerrymandering on voters?
        \item \textbf{Bumper Sticker:} Significant gerrymandering inspires good feelings as it increases the chances that you are represented by a co-partisan. It, can, however, be in conflict with your democratic values. This means that when primed with political conflict frames, those living in gerrymandered districts that help co-partisans win elections, will feel better about it and those not represented by a co-partisan will feel much worse about it. When these frames are not present, however, people may report lower levels of support for gerrymandering regardless of whether they are represented by a co-partisan or out-partisan.
        \item \textbf{Sell:} Gerrymandering gets a lot of blame as an elite-driven source of political polarization. However, I believe that voters, generally, are actually more supportive of gerrymandering than they like to report in surveys. In surveys, they are likely to be thinking of democracy and have been taught that it has a negative connotation in the context of democratic quality. However, partisan teamsmanship is quite strong. When we are thinking in terms of competition, I expect that these concerns about democratic quality fly out the window and support for gerrymandering increases for those represented by co-partisans and decreases fro those represented by out-partisans. Furthermore, when having these valanced affects, respondents will report differing levels of trust in the political system and will feel less efficacious when faced with this information.
        \item \textbf{Chapter Outline:}
        \begin{enumerate}
            \item Introduction; history of partisan gerrymandering, how it is believed to effect polarization from structural perspectives
            \item Detecting partisan gerrymandering; how to measure it
            \item Do people report more polarized views of gerrymandering when primed to think of political competition than when primed to think of democratic quality?
            \item What does this information do to levels of trust and efficacy among those represented by co-and-out-partisans?
            \item Feelings of Negative partisanship and efficacy as a political minority - how does living in a district where you are a political minority increase levels of distrust, feelings that the other party is corrupt and incompetent, and increase levels of negative partisanship?
        \end{enumerate}
        \item \textbf{Challenges or shortcomings:} I honestly don't know how this would be received. Would it be a 'no duh' response? Or 'cool, gonna cite this'? What can I do to really add some shock and awe to this? Could include a deliberation chapter maybe, I could imagine I could do some cool measurement stuff here which might help. Just not fully sure...
        \item \textbf{How excited am I about this idea?} $\star$ $\star$ $\star$
    \end{itemize}
\subsection{Partisan Slang}
    \begin{itemize}
        \item \textbf{Question:} Do partisans use distinctive terms and phrases? Do they use slang?
        \item \textbf{Bumper Sticker:} Slang is (1) a communicative tool, (2) used for inclusion, (3) used for exclusion. Can slang be used by partisans to get at these three components of slang?
        \item \textbf{Sell:} I haven't seen all that much about the use of party-specific slang and what it can do to being a type of jargon to communicate ideology, and to indicate group affiliation
        \item \textbf{Chapter Outline:}
        \begin{enumerate}
            \item Introduction; What is slang from a psycho-linguistic perspective? What is it used for and how can it be important for understanding partisanship?
            \item Can we detect slang? How prevalent is it in political discourse?
            \item How distinct is the language that partisans use when looking at slang?
            \item Is slang a source of ideological information?
            \item Does slang provide information for inclusion and exclusion of individuals?
            \item Has the alt-right and far left contributed to the slang used in the main stream media? What does this tell us about their seemingly increasing influence in political discourse?
        \end{enumerate}
        \item \textbf{Challenges or shortcomings:} What slang, more generally, is used for is pretty well-covered in linguistics and social psychology. Measuring this specific type of slang may be a bit tricky, though. I would need to see how linguists do it for various groups. I am assuming it is possible.
        \item \textbf{How excited am I about this idea?} $\star$ $\star$ $\star$
    \end{itemize}
\subsection{Bye-bye good-natured political discussions: How elites have shaped norms around political discussions among the public}
    \begin{itemize}
        \item \textbf{Question:} Do we still have political discussions that recognize equal status between groups, that share a common goal, encourage between-group cooperation, and the presence of institutional support for cooperation? Have these features left the goals of the public as a result of the media? What motivations are encouraged by such models of conversations?
        \item \textbf{Bumper Sticker:} Measures in the ANES show that people do have political conversations with those that we disagree with. However, these measures do not tell us the quality of those conversations. This means that we should be cautious of assuming the quantity is equivalent to the quality of the content of those conversations. Some scholars have suggested that we might reduce political polarization through having conversations with those we disagree with. However, this happens under a set of particular features of the conversation. When these features are not present, we cannot guarantee that these conversations have these positive effects of reducing political polarization. Scholarly attention on the media's role in influencing political discussions, seem to indicate that the media encourage the exact opposite behavior and goals behind a political discussion with an out-partisan than what is needed to have these positive discussions. As a result, I think that we have a problem where deliberation is not going to be as effective for reducing polarization as some claim.
        \item \textbf{Sell:} Work on social influence has been seeing more serious attention lately. However, common survey measures do not tell us all that much about the quality of conversation but rather just the structure of it. From these measures, scholars have assumed that polarization has the chance to reduce polarization. Some scholars have wanted to dig into this claim by using tools meant to measure affect like skin conduction. Discussions with out-partisans \textit{can} help with learning alternative viewpoints and reduce polarization, but under certain circumstances where the goals of the participants of the conversation line-up with these outcomes. However, it is well documented that the media do not provide a great model for engaging in political discussion nor does it encourage people to hold such goals. We should dig more into descriptively how common does the media encourage such behavior, how common it is for the public to engage in political discussions that have these features leading to positive reductions in polarization, and whether the media is a direct source for teaching the public how to have conversations with those you disagree with.
        \item \textbf{Chapter Outline:}
        \begin{enumerate}
            \item Introduction; what does a conversation need to be to reduce polarization?
            \item Do modal debates on network television encourage accuracy or group-based motivations?
            \item Does the media provide a model for how to engage in de-polarizing discussions?
            \item Do discussions among the public contain these features for de-polarization discussions?
            \item Does intervention by media encouraging conversations with and without these features effect levels of polarization?
            \item Conclusions; implications
        \end{enumerate}
        \item \textbf{Challenges or shortcomings:} Somewhat derivative of Mutz's work so I need to do a bit more thinking to find the right angle; If I can, I'd have tons of fun with this one.
        \item \textbf{How excited am I about this idea?} $\star$ $\star$ $\star$
    \end{itemize}
\subsection{Fear or Hatred: Is political polarization the result of fear or of hatred of out-partisans?}
    \begin{itemize}
        \item \textbf{Question:} Is political polarization the result of fear or of hatred of out-partisans?
        \item \textbf{Bumper Sticker:} Negative partisanship is not necessarily just anger and hatred of out-partisans, but it may also present itself as fear of the other group as a result of the common message that out-partisans are dangerous to one's way of life.
        \item \textbf{Sell:} Affective polarization has become a super popular theory explaining where political polarization comes from in the public. The measurement of it, however, is not as convincing. Negative partisanship and affective polarization are often pitched as the result of hatred or anger of the other group. However, elite messages to co-partisans often emphasize the need to fear the values and goals of out-partisans. The common individual-level survey measure of affective polarization is rather agnostic between the particular affect driving these differences between feeling thermometer scores a respondent assigns to their party and the out-party.
        \item \textbf{Chapter Outline:}
        \begin{enumerate}
            \item Introduction; how is affective polarization characterized w.r.t. emotions?
            \item Fear and the prevalence of it in media messages about out-partisans
            \item Messages encouraging fear, hatred, and anger towards out-partisans on 
            \item Detecting affective polarization as consequence of fear and compared to hatred and anger
            \item Prevalence of affective polarization with its new measure
            \item Conclusions; implications
        \end{enumerate}
        \item \textbf{Challenges or shortcomings:} I would probably need to spend a decent amount of time working through measurement. Self-reports have a hard time distinguishing between these emotions. I know that there is some work out there on this like using the PNAS and PNAS-M scales...
        \item \textbf{How excited am I about this idea?} $\star$ $\star$ $\star$
    \end{itemize}
\subsection{America, you are condescending: How the American pastime of looking down on the poor and working class effects the public opinion's feedback on policy}
    \begin{itemize}
        \item \textbf{Question:} Can we detect an air of condescension toward the poor in public opinion? What does this do feedback on policy that fails to help the poor and working class?
        \item \textbf{Bumper Sticker:} The public and elites are both condescending toward the poor and working class in the United States. This condescension has significant impacts on class-related policy in the United States. It makes policy that expands welfare and regulates inequality unresponsive and rather unpopular.
        \item \textbf{Sell:} While some have made the argument that public opinion tends to express sympathy for the poor and that elites who show strong bias in helping the wealthy tend to experience electoral punishment, there does not seem to be much observable evidence for this argument. I argue that though survey responses provide evidence suggesting sympathy expressed toward the poor, I argue that it is not sympathy or resentment we should be capturing in these survey prompts, but rather we should be looking for condescension. We see in attitudes toward welfare and race, that White Americans tend to see welfare policies that are framed to help Black Americans as unpopular. Common arguments against such policies are linked to responses expressing racial resentment - all of which contain some sort of moral judgement about deservingness of such support. While, there is certainly race-based heterogeneity in willingness to support polices meant to help the poor, I believe that existing measures getting at preferences of helping the poor and working class writ-large, fail to capture similar moral judgement about deservingness. 
        \item \textbf{Chapter Outline:}
        \begin{enumerate}
            \item Introduction; the debates of whether the public hate the poor or hate the wealthy
            \item Broadening what attitudes we think the public can hold toward the poor. Is it sympathy, resentment, condescension, pity? Is it that they think they get more than they deserve, or should we focus on attitudes of what we think they do with the money they have that drives public opinion?
            \item What attitudes do the news media hold against the poor? How prevalent are arguments of work ethic used when discussing the poor?
            \item Do the public parrot any of these arguments toward the poor?
            \item What does this mean about willingness to accept frames from politicians that downplay the relevance of class?
        \end{enumerate}
        \item \textbf{Challenges or shortcomings:} I'd definitely be contradicting Spencer Piston's work - which I like. But generally, I just kind of disagree with how we currently get at attitudes of the public toward class - I think we don't have very convincing arguments for the paradoxical nature of what surveys tell us versus what happens when it comes to actual policy and the debates that are had. I also think that, conceptually, we are looking at the wrong thing when trying to measure the public's attitudes toward class. I feel that it is pretty hard to deny that many poor Americans are making less money than they should - particularly when people are primed to think about the stagnation of the minimum wage. I think where a lot of the public's attitudes come from in relation to issues of class is not what they make, but what they do with what they make. I think people often hold more negative stereotypes, not just based in race, about what poor people spend their money on - there seems to be quite a bit of a moral superiority baked into policy, debates, and public opinion where we feel that we can evaluate what people choose to do with their money and we then evaluate whether we think those are ``right" choices.
        \item \textbf{How excited am I about this idea?} $\star$ $\star$ $\star$
    \end{itemize}
\subsection{Dude, stop being so stubborn: Conceptualizing partisan polarization as stubbornness}
    \begin{itemize}
        \item \textbf{Question:} Can we re-conceptualize the tendency for partisans to not listen to alternative view points as stubbornness?
        \item \textbf{Bumper Sticker:} Yes, we can. This also gives us some opportunities to explore other ways in which we might be able to persuade people based on clinical psychologists' understanding of stubbornness potentially.
        \item \textbf{Sell:} Stubbornness can be seen as motivated by a strong desire to not be wrong, punished, or to have their identity doubted. This is definitely what we see among partisans in contemporary American politics - strong partisans do not like and do not listen to conflicting information. Clinical psychologists have a relatively deep understanding of what stubbornness is and what we can do to help reduce it. As we as a subfield are asking ourselves: ``what can we do to persuade these polarized partisans?", conceptualizing polarization as stubbornness, we may gain some insights about how we might reduce it in an ideal world. Emphasis on an ideal world.
        \item \textbf{Chapter Outline:}
        \begin{enumerate}
            \item Introduction; the current state of political polarization; what is stubbornness as a psychological concept?
            \item How might we think of political polarization as a form of stubbornness?
            \item What does this reconceptualization do to our understanding of where polarization comes from?
            \item How can we reduce polarization if we conceptualize it as stubbornness?
            \item Conclusion and implications
        \end{enumerate}
        \item \textbf{Challenges or shortcomings:} Is it unique or interesting enough? Perhaps. I think what will depend here is whether I can spend significant time thinking about the right frame to convince people that this is a useful way to think of it. I think generally we see polarization \textit{kind of} as stubbornness, but I think putting emphasis on explicitly tying it to the way that psychologists conceptualize it can be helpful - and I would need to be sure I clearly make that argument.
        \item \textbf{How excited am I about this idea?} $\star$ $\star$ $\star$ - We really are kind of trying to figure out what do we do with polarization. We know it happens and we know that there are a lot of mechanisms encouraging more of it. But there seems to be a bit of this nihilism around how we try to reduce it. I think creatively using different concepts from our peers in psychology can help us push the needle forward.
    \end{itemize}
\subsection{Perceptions of ideological extremity and roles within the party for female politicians}
    \begin{itemize}
        \item \textbf{Question:} What do the public expect of female politicians?
        \item \textbf{Bumper Sticker:} The public likely expect female politicians to be better party soldiers and, in times of polarization, this means that once they have successfully entered office, to be more ideologically extreme to the right or left.
        \item \textbf{Sell:} Classic work on the role of sex and gender in perceptions of candidates seem to indicate that, in many cases, female candidates are assumed to be a bit more liberal than their male co-partisan competitors. Some more recent work suggests that Republican voters in low-information environments have a bit harder time agreeing about the ideological position of Republican female candidates. Despite these indications that we should expect Republican females to be a bit more moderate and female Democratic candidates to be more liberal, it appears that there is this movement among female politicians on both the left and right to be more ideologically polarizing. Connecting this with gender stereotypes about being agreeable, in an age of polarization, the public may evaluate their female representatives on the success by which they act as good party soldiers. Specifically, I expect this to be most prominent among Republican voters. While I expect this to work for Democratic voters as well, I think in line with work from the 90's, this will be more rooted in gender-traits as opposed to beliefs.
        \item \textbf{Chapter Outline:}
        \begin{enumerate}
            \item Introduction; classic understandings linking sex and gender to ideology.
            \item Presenting the argument; linking gender stereotypes, ideology, and expectations for legislative style among constituents
            \item Descriptively, what are common beliefs about female politicians among partisans?
            \item Are party soldiers seen as more ideologically polarized by the public?
            \item Do Republicans and Democrats expect their female representatives to take on similar roles?
            \item Conclusions; implications
        \end{enumerate}
        \item \textbf{Challenges or shortcomings:} Is this the right angle I should be taking from this anecdotal evidence I've been collecting? Not sure if I am on the right track on how to explain the AOC, Boebert, MJT types and whether they are outliers or if Susan Collins and Nancy Pelosi are the norm. Is this an age thing and not gender thing?
        \item \textbf{How excited am I about this idea?} $\star$ $\star$ $\star$
    \end{itemize}
\subsection{Do gendered norms about masculinity come from norms surrounding politics?}
    \begin{itemize}
        \item \textbf{Question:} Do people link politics and masculinity as a result of norms communicated through elite messaging or through psychological concerns about competence at handling threats (e.g. war and crime)?
        \item \textbf{Bumper Sticker:} Men often are attributed more with instrumental traits rather than the communal ones that women are often associated with. Is this link between gender and sex the result of elite messaging, or is it taking advantage of more base stereotypes coming from  earlier human history of masculine characters helping reduce concerns about threat? I honestly do think it comes from elite messaging. Yes, while some have attributed preferences for masculinity to politics during times of war because of these more historically functional perspectives, I think that most of the time preferences for masculinity just come from the media's penchants for associating it with success even outside of times of external threats.
        \item \textbf{Sell:} There is a lot of amazing work on gender and sex in politics. One thing that isn't resolved for me, though, is where exactly these long-lasting preferences for masculinity comes from. My read of the literature is that, voters had been seen as preferring men more, then in the shift to supply narratives of emergence the literature argued that candidates with more masculine traits overestimated their qualifications and were more likely to run and were then more prevalent, more recent work is arguing that we have more women running but these women have more masculine traits and that women and men with more feminine traits are less likely to emerge as candidates. Out of all this, it tells me that there is this latent and long-lasting preference for masculinity, but why is that the case? There are two competing theories I've seen one that use a more human evolution sort of spin on it that masculine leaders were preferred in times of crisis. The other would be that it is through elite messaging that we continually hear messages linking masculinity with political success.
        \item \textbf{Chapter Outline:}
        \begin{enumerate}
            \item Introduction; preferences for masculinity and the key theories for why
            \item Why does elite messaging matter?
            \item How common are the linkages between masculine traits and positive political outcomes made?
            \item Do the public respond to messages highlighting the positive political outcomes of feminine traits? Is it dependent on ideology?
            \item Conclusions and implications
        \end{enumerate}
        \item \textbf{Challenges or shortcomings:} Is there something I am missing in the literature that has already covered this? I don't think I have seen anything, but I don't want to be stepping on toes. Could I make a treatment strong enough to override years and years of messages linking masculine traits to positive political outcomes?
        \item \textbf{How excited am I about this idea?} $\star$ $\star$ $\star$
    \end{itemize}

% Two Stars 
\section{Two Stars}
\subsection{A participatory model of local recall elections}
    \begin{itemize}
        \item \textbf{Question:} Who participates in local recall elections?
        \item \textbf{Bumper Sticker:} N/A; not sure really...
        \item \textbf{Sell:} Models of sub-national political participation are somewhat incomplete. Existing models are quite dependent on national models of participation and there is a lot of open space to examine participation at the sub-national level. Further, our understanding how recall elections work seems to be rather sparse. Understanding who participates in recall elections not only shines a light on modeling sub-national political participation but might help us further understand the goal of recall elections for those who are following them.
        \item \textbf{Chapter Outline:}
        \begin{enumerate}
            \item Introduction; why are recall elections important and what do we know about them
            \item What do we know about models of sub-national political participation and what makes recall elections different
            \item Data collection process and descriptive understanding of who participates in these elections
            \item Providing a causal model of what factors matter for predicting participation in local recall elections
            \item Conclusions
        \end{enumerate}
        \item \textbf{Challenges or shortcomings:} Data collection will be tricky here. This also is kind of a niche thing so it will make it a bit of a tough sell in terms of its significance and building a theory might be a bit tricky.
        \item \textbf{How excited am I about this idea?} $\star$ $\star$
    \end{itemize}
\subsection{Positive Emotions and its effect on polarization}
    \begin{itemize}
        \item \textbf{Question:} Do positive emotions lower the effects of political polarization?
        \item \textbf{Bumper Sticker:} Political polarization is often associated with negative emotions. The literature on emotions in politics is quite sparse in terms of its understanding of the relevance of positive emotions in politics. Can positive emotions reduce differences in in-group and out-group preferences? Do they asymmetrically contribute to them by increasing positive in-group attitudes?
        \item \textbf{Sell:} The literature on emotions in politics are relatively scarce in its examination of positive emotions in politics. The literature on polarization evokes the terminology of affective polarization, which some believe is where the public currently are at, but these seen to be relevant emotions are almost always negative. Polarization is not just the result of negative out-group attitudes but also positive in-group attitudes. Do positive attitudes contribute to the difference by solidifying positive in-group attitudes?
        \item \textbf{Chapter Outline:}
        \begin{enumerate}
            \item Affective polarization and negative emotions
            \item The role of positive emotions in politics and in contributing to differences in in-group and out-group attitudes
            \item How common are positive emotions in politics?
            \item Do positive emotions encourage increases in polarization? Is it because of its effects on increasing in-group attitudes?
        \end{enumerate}
        \item \textbf{Challenges or shortcomings:} Positive emotions are not super well understood in the literature so I'd be balancing quite a few spinning plates. Measurement would also be relatively tricky as well.
        \item \textbf{How excited am I about this idea?} $\star$ $\star$ - kind of derivative of my real interest in the first two ideas.
    \end{itemize}
\subsection{Gosh, criticism is so hard: How the attention requirements of criticism limit's the public's tendency to critique policy}
    \begin{itemize}
        \item \textbf{Question:} Are there ceiling effects for the amount of criticism that politicians will face from the public in the aggregate?
        \item \textbf{Bumper Sticker:} We already know that the degree to which a politician or set of politicians is going to face backlash for policy decisions is dependent on partisanship. The other question, which is much more rooted in traditional democratic theory, is whether the attention and engagement that the public tend to be willing to give is sufficient enough to put enough pressure on political elites to change course on policymaking.
        \item \textbf{Sell:} Political scientists, particularly those interested in participation have long been asking this question from as many angles as they can come up with. What remains unclear is if we just directly test at what point is there an ``attention tipping point" where we can get the public to actually start paying enough attention to pose real critiques of policy. 
        \item \textbf{Chapter Outline:}
        \begin{enumerate}
            \item 
        \end{enumerate}
        \item \textbf{Challenges or shortcomings:} Bleh. This feels a bit derivative as it stands. I should take time to read and think more about this idea to flush it out some more.
        \item \textbf{How excited am I about this idea?} $\star$ $\star$
    \end{itemize}

% One Star
\section{One Star}
\subsection{Cancel Culture: Next form of racism?}
    \begin{itemize}
        \item \textbf{Question:} Is cancel culture just the next form of racism? Is there heterogeneity in how people conceptualize it? What is it?
        \item \textbf{Bumper Sticker:} There might be an argument to be made that cancel culture is just the next step in overt racism $\rightarrow$ symbolic and color-blind racism $\rightarrow$ dominant group victimization
        \item \textbf{Sell:} This is a pretty hot topic among contemporary Republicans and modern conservatives. Might it just be a new way to express racist attitudes while trying to hold some ideological legitimacy?
        \item \textbf{Chapter Outline:}
        \begin{enumerate}
            \item What is cancel culture? How and under what conditions is it used?
            \item What is it as an ideological position?
            \item Is it just a defense for being called out for expressing racist views?
            \item Is it used to communicate victimhood for racially conservative whites?
        \end{enumerate}
        \item \textbf{Challenges or shortcomings:} It is hard to try to conceptualize something that might just be used in a ideologically inconsistent way. If it is just a term used to spark a particular reaction without communicating any meaningful ideological position, then understanding what it is might not be difficult but perhaps futile.
        \item \textbf{How excited am I about this idea?} $\star$
    \end{itemize}
\subsection{The Fanatic Republicans and Self-Loathing Democrats: How personality shapes partisan stereotypes}
    \begin{itemize}
        \item \textbf{Question:} Is it a personality type among Republicans to be fanatic party soldiers while Democrats are much more self-critical? Is this just because of other attributes that they hold such as education and tendency to be reactive to particular stimuli?
        \item \textbf{Bumper Sticker:} Honestly I need to do much more reading about this. In particular, I should read Matt Grossman's book.
        \item \textbf{Sell:} I think there is a lot of interesting work to be done on the role that predispositions have on partisanship and how that explains the perfect mixture of nature and nurture leading to the polarization we see today.
        \item \textbf{Chapter Outline:}
        \begin{enumerate}
            \item
        \end{enumerate}
        \item \textbf{Challenges or shortcomings:} Honestly, this is a bit of a meh idea. It is something I put down when I was thinking about partisan stereotypes. I honestly haven't had enough time to sit down and read more on this sort of stuff to come up with something that I think could be good. But, generally, continuing to look at these tendencies for both elites and the public to react to their own performance in governance is kind of interesting; I wonder if there is an angle that someone else might see that could be make it super worth it to jump into this.
        \item \textbf{How excited am I about this idea?}: $\star$
    \end{itemize}
\subsection{Thermostatic Public Opinion or Thermostatic Survey Response?: How losing effects survey responses}
    \begin{itemize}
        \item \textbf{Question:} Are macropartisan shifts the result of thermostatic public opinion or is it simply the samples and their responses we get?
        \item \textbf{Bumper Sticker:} Macropartisanship tells us that though partisanship is sticky, there are substantive shifts among some members of the public in the aggregate. With recent challenges in pollster's ability to get reliable predictions of electoral outcomes from polls from individual level analyses, one potential explanation may be that we should examine aggregate trends to make inferences about immediate public opinion shifts after a lost election.
        \item \textbf{Sell:} There is lots of debate about whether there is thermostatic public opinion or whether it is just an artifact of bias coming from survey error. Taking a deep dive to try to understand this better in the aftermath of a lost election might prove useful to understanding survey methodology, reactions to losing in elections, and macropartisanship.
        \item \textbf{Chapter Outline:}
        \begin{enumerate}
            \item
        \end{enumerate}
        \item \textbf{Challenges or shortcomings:} Honestly, I need to think quite a bit more about this. There is a lot of really cool work being done intersecting with following aggregate PO over time - both as a substantive and methodological literature. But here, I am thinking a lot along the lines of trying to use aggregate PO data to help with some of these debates happening among election forecasters with whether poor model performance for 538 was because of poor sampling or ``survey trolling" among Republicans.
        \item \textbf{How excited am I about this idea?} $\star$
    \end{itemize}
\subsection{Lessons learned from labor: How the great resignation teaches the public about the efficacy of collective participatory democracy}
    \begin{itemize}
        \item \textbf{Question:} With a lack of labor movement in the US, does the great resignation happening in the early 2020's help introduce Americans not just to class politics but is it a demonstration of the efficacy of collective participation in a democratic process? That is, do those who participate in the great resignation become more likely to participate in political participation as a result?
        \item \textbf{Bumper Sticker:} The great resignation can be a source of education about collective participation and be a source of efficacy for participating in these types of behaviors. Because of this, this may then be applied to the efficacy of democratic political participation.
        \item \textbf{Sell:} The US is notorious for being relatively low in terms of propensity to engage in a labor movement and to see politics as a form of class conflict. However, a powerful collective movement by employees can provide information about the efficacy of a collective movement to express voice to the powerful. As it is much more widespread and a bit less ideological than the Occupy movement in the late 2000's and 2010's, it may be a chance for a more universal form of education and growth in efficacy for participatory politics.
        \item \textbf{Chapter Outline:}
        \begin{enumerate}
            \item Introduction; history of the labor movement in the US
            \item How the history of the labor movement has effected participatory politics in the US and how the great resignation may be a way to increase participatory efficacy among those who are part of it
            \item Whether people learned about collective participation from the great resignation
            \item Whether people felt more efficacy as a result of participating in it
            \item Whether we can link it to political participation as a causal factor
            \item Conclusions and implications
        \end{enumerate}
        \item \textbf{Challenges or shortcomings:} It feels like a very meh idea. Hard to sell myself on; so very hard to sell others on I feel like. Anticipate it would be hard for me to stay excited about for the long-haul.
        \item \textbf{How excited am I about this idea?} $\star$
    \end{itemize}


\begin{comment}
\subsection{}
    \begin{itemize}
        \item \textbf{Question:}
        \item \textbf{Bumper Sticker:}
        \item \textbf{Sell:}
        \item \textbf{Chapter Outline:}
        \begin{enumerate}
            \item
        \end{enumerate}
        \item \textbf{Challenges or shortcomings:}
        \item \textbf{How excited am I about this idea?}
    \end{itemize}
\end{comment}
\end{document}